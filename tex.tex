\documentclass{article}

% Language setting
% Replace `english' with e.g. `spanish' to change the document language
\usepackage[english, russian]{babel}
\usepackage{listings}
% Set page size and margins
% Replace `letterpaper' with `a4paper' for UK/EU standard size
\usepackage[letterpaper,top=2cm,bottom=2cm,left=3cm,right=3cm,marginparwidth=1.75cm]{geometry}
\usepackage{amsmath,amsfonts,amssymb,amsthm,mathtools}

% Useful packages
\usepackage{amsmath}
\usepackage{graphicx}
\usepackage[colorlinks=true, allcolors=blue]{hyperref}



\title{Глобальные градиентные методы оптимизации}
\author{Судаков Илья}

\begin{document}
    \maketitle



    \newpage


    \section{Градиентный спуск}
    Пусть целевая функция имеет вид:
    $F(X): X \rightarrow R$\\\\
    Задача оптимизации задана в следующем виде:
    $F(X) \xrightarrow{} min$\\\\
    Основная идея метода заключается в том, чтобы идти в направлении наискорейшего спуска, а это направление задаётся антиградиентом $-\nabla F$\\\\
    и $X^{[i+1]}=X^{[i]}-\lambda^{[i]}\nabla F(X^{[i]})$
    где $\lambda ^{[j]}$ задает скорость градиентного спуска и может быть выбрана как
    \begin{itemize}
        \item постоянная, тогда метод может не сходиться
        \item убывать по какому-то закону
        \item гарантировать наискорейший спуск
    \end{itemize}

    \subsection{Алгоритм градиентного спуска}
    \begin{itemize}
        \item Задают начальное приближение и точность расчёта: $X_0, \varepsilon$
        \item Рассчитывают $X^{[i+1]}=X^{[i]}-\lambda^{[i]}\nabla F(X^{[i]})$
        \item Проверяют условие остановки (на усмотрение):
        \begin{itemize}
            \item $|x^{[j]}-x^{[j+1]}|<\varepsilon$
            \item $|F(x^{[j]})-F(x^{[j+1]})|
            <\varepsilon$
            \item $\nabla F(X^{[i]})<\varepsilon$
            \item иначе переходят к пункту 2
        \end{itemize}
    \end{itemize}

    Этот алгоритм будет реализован на python

    \subsection{Метод наискорейшего спуска}
    В случае наискорейшего спуска $\lambda^{[j]}$ определяется как:\\
    \begin{gather*}
        \lambda^{[j]}=argmin F(X^{[j+1]}) = argmin F(X^{[j]} - \lambda\nabla F(X^{[i]}))
    \end{gather*}
    чтобы вычислить $\lambda^{[j]}$ будем использовать метод золотого сечения




































    \newpage


    \section{Метод золотого сечения}

    \subsection{Описание}

    Метод золотого сечения — это эффективная реализации троичного поиска, служащего для нахождения минимума/максимума унимодальной функции на отрезке. Лучше тем, что на каждой итерации вычисляется значение в одной, а не двух точках.\\\\
    Пусть хотим найти минимум на отрезке $[a, b]$, тогда разобьем его на 3 части 2 точками $s_1, s_2$, так чтобы при отсекании одного из крайних подотрезков, одна из точек осталась границей подотрезков, а соотношение длин подотрезков оставалось прежним.\\
    $|[a, s_1]|=|[s_2, b]|=l_1$\\
    $|[s_1, s_2]|=l_2$\\
    тогда: $\frac{l_2}{l_1}=\frac{l_1-l_2}{l_2} \Rightarrow l_1^2-l_1\cdot{l_2}-l_2^2=0 \Rightarrow \frac{l_1}{l_2}=\frac{1+\sqrt{5}}{2}=\phi$

    \subsection{Алгоритм тернарного поиска}
    написать ли

    \subsection{Оценка сложности}
    Так как на каждой итерации длина рассматриваемого отрезка умножается на $\frac{l_1+l_2}{l_1+l_2+l_2}=\frac{3+\sqrt{5}}{4+2\sqrt{5}}=\phi^{-1}$, то для достижения точности $\delta$ потребуется $\frac{\ln(\frac{|[a, b]|}{\delta})}{\ln(\phi)} \approx 2\ln(\frac{|[a, b]|}{\delta})$ итераций.

    \subsection{Применение}
    А что если функция не унимодальная, тогда к чему сойдется этот метод? К точке, которая является одним из локальных минимумов функции либо одному из концов, в котором производная неположительная.\\

    \newpage


    \section{Глобальная одномерная оптимизация}
    Алгоритм поиска глобального минимума функции вдоль одного направления будет основываться на интервальном анализе. Перечислим основные определения и теоремы, которые нам пригодятся для посторения алгоритма.

    \subsection{Определение}
    Интервалом $[a, b]$ называется следующее множество:
    \begin{gather*}
    [a,b]
        :=\{x \in \mathbb{R} | a \le x \le b\}
    \end{gather*}

    \subsection{Арифметические свойства интервалов}
    \begin{itemize}

        \item {$\bold{x} + \bold{y} =$ {${\bold{\underline{x}} + \bold{\underline{y}},  \bold{\overline{x}} + \bold{\overline{y}}}$}}

        \item {$\bold{x} - \bold{y} =$ {${\bold{\underline{x}} - \bold{\overline{y}},  \bold{\overline{x}} - \bold{\underline{y}}}$}}

        \item {$\bold{x} \cdot \bold{y} =$ {${min(\bold{\underline{x} \underline{y}}, \bold{\underline{x} \overline{y}}, \bold{\overline{x} \underline{y}}, \bold{\overline{x} \overline{y}}) , max(\bold{\underline{x} \underline{y}}, \bold{\underline{x} \overline{y}}, \bold{\overline{x} \underline{y}}, \bold{\overline{x} \overline{y}}) }$}}

        \item {$\bold{x} \div \bold{y} =$ $\bold{x} \cdot[1/\bold{\overline{y}}, 1/\bold{\underline{y}}]$, если $0 \not\in y$}

    \end{itemize}

    а как быть с другими функциями

    \subsection{Основная теорема интервальной арифметики}
    Пусть $f(x_1, ..., x_n)$ - рациональная функция вещественных аргументов $x_1, ..., x_n$, и для нее определен результат $\bold{F(X_1, ..., X_n)}$ подстановки вместо аргументов интервалов их изменения $(X_1, ..., X_n) \subset \mathbb{R}^n$ и для $(X_1, ..., X_n)$ операциии выполняются по правилам интервальной арифметики. Тогда
    \begin{gather*}
        \{f\{x_1,...,x_n\} | x_1 \in \bold{X_1},...,x_n \in \bold{X_n}\} \subset \bold{F(\bold{X_1},...,\bold{X_n}})
    \end{gather*}

    как лучше преобразовать выражение чтобы функция быстрее сходилась

    \subsection{Утверждение}

    Пусть $f(x_1, ..., x_n)$ - рациональная функция вещественных аргументов $x_1, ..., x_n$, и $\bold{F(X_1, ..., X_n)}$ соответствующая ей интервальная функция, тогда выполняется монотонность по включению.
    Пусть $\bold{X_1},...,\bold{X_n}$ и $\bold{Y_1},...,\bold{Y_n}$, такие что $\bold{X_1} \subset \bold{Y_1},...,\bold{X_n} \subset \bold{Y_n}$, тогда
    \begin{gather*}
        f\{\bold{X_1},...,\bold{X_n}\} \subset f\{\bold{Y_1},...,\bold{Y_n}\}
    \end{gather*}

    \subsection{Определение}

    Пусть $N > 0$  натуральное число. Тогда если $\langle S_0,..., S_{n-1} \rangle$ семейство непустых подмножеств множества $S$, тогда будем называть его разбиением множества $S$. В частности, обьединение $S_0,..., S_{n-1}$ равно $S$ , тогда последовательность $\langle S_0,..., S_{n-1} \rangle$ является покрытием $S$ .

    \subsection{Теорема}

    Рассмотрим задачу безусовной глобальной оптимизации. Пусть $\langle B_0,..., B_{n-1} \rangle$ семейство множеств, содержащее глобальный минимум, такое что упорядочено по возрастанию нижней границы $f(B_i)$ для $i=0, 1, 2,..., N-1$. Пусть $U$ наименьшая из верхних значений функции для подмножеств $\langle f(B_0),..., f(B_{n-1}) \rangle$. Тогда интервал $[lb(f(B_0)), U]$ содержит глобальным минимум $\mu$.


    используя это напишем алгоритм поиска глобального минимума вдоль направления


    \newpage
    \bibliographystyle{alpha}
    \bibliography{sample}




    Многие методы оптимизации (координатный спуск, градиентный спуск, метод сопряженного гадиента и др.) используют так называемый локальный поиск, т.е. поиск минимума одномерной функции вдоль луча в n-мерном евклидовом пространстве. При этом всегда находится локальный минимум. В проекте предлагается исследовать потенциал применения методов глобальной одномерной оптимизации для организации линейного поиска. В каких случаях удастся найти глобальной минимум многомерной функции, используя глобальный линейный поиск в контексте классических градиентных методов.


    1. Программная реализация методов глобального линейного поиска. 2. Автоматическая проекция функции на направление с использованием пакета sympy. 3. Реализация алгоритма оптимизации многомерной функции. 4. Экспериментальное и теоретическое исследование.


    "1. Алгоритм поиска глобального минимума одномерной функции.
    2. Алгоритм оптимизации многомерной функции, основанный на комбинировании классического градиентного подхода и глобального линейного поиска.
    3. Программная реализация алгоритма на языке Python.
    4. Результаты вычислительных экспериментов.
    5. Возможно написание статьи по результатам проекта."
\end{document}