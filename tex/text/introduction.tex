\subsection{Мотивация}
    Многие методы оптимизации (координатный спуск, градиентный спуск, метод сопряженного гадиента и др.) используют так называемый локальный поиск, т.е. поиск минимума одномерной функции вдоль луча в n-мерном евклидовом пространстве. При этом всегда находится локальный минимум. В проекте предлагается исследовать потенциал применения методов глобальной одномерной оптимизации для организации линейного поиска. В каких случаях удастся найти глобальной минимум многомерной функции, используя глобальный линейный поиск в контексте классических градиентных методов.

    \subsection{Задачи проекта}
    \begin{itemize}
        \item Программная реализация методов глобального линейного поиска.
        \begin{itemize}
            \item Метод Золотого сечения
            \item Алгоритм Moore-Skelboe
        \end{itemize}
        \item Реализация алгоритма оптимизации многомерной функции.
        \begin{itemize}
            \item Координатный спуск
            \item Градиентный спуск
        \end{itemize}
        \item Экспериментальное и теоретическое исследование.
    \end{itemize}

    \subsection{Цели проекта}
    Написать алгоритм, который будет находить глобальный минимум многомерной непрерывной функции:
    \begin{itemize}
        \item с хорошей скоростью
        \item с заданной точностью
        \item не хуже классических методов для унимодальных функций
    \end{itemize}

    \subsection{Ссылка на материалы}
    Ссылка на репозиторий, с разработанными материалами:\\
    \href{https://github.com/sudakovcom/global-optimization.git}{https://github.com/sudakovcom/global-optimization.git}